
 %%%%%%%%%%%%%%%%%%%%%%%%%%%%%%%%
%%%%%%%%%%%%%%%%%%%%%%%%%%%%%%%%%%
%%        LATEX TEMPLATE        %%
%%                              %%
%%     By: Sammy Haq            %%
%% e-mail: sammy.haq1@gmail.com %%
%%                              %%
%%%%%%%%%%%%%%%%%%%%%%%%%%%%%%%%%%
 %%%%%%%%%%%%%%%%%%%%%%%%%%%%%%%%


 %%%%%%%%%%%%%%%%%%%%%
% DOCUMENT FORMATTING %%%%%%%%%%%%%%%%%%%%%%%%%%%%%%%%%%%%%%%%%%%%%%%%%%%%%%%%%
 %%%%%%%%%%%%%%%%%%%%%

\documentclass[12pt, letterpaper]{article}
\usepackage[utf8]{inputenc}
\usepackage{indentfirst} % Sets the first paragraph to always indent first.


 %%%%%%%%%%
% PACKAGES %%%%%%%%%%%%%%%%%%%%%%%%%%%%%%%%%%%%%%%%%%%%%%%%%%%%%%%%%%%%%%%%%%%%
 %%%%%%%%%%

%%%%%%%%%%%%%%%%%%%%%%%%%%%%%%%%%%%%%%%%%%%%%%%%
% For Advanced Mathematical Operation Things. %
%%%%%%%%%%%%%%%%%%%%%%%%%%%%%%%%%%%%%%%%%%%%%%

\usepackage{amsmath} % ABSOLUTELY NECESSARY FOR BASICALLY EVERYTHING MATH
\allowdisplaybreaks % allows environments like align to span multiple pages

%%%%%%%%%%%%%%%%%%%%%
% PDF Integration. %
%%%%%%%%%%%%%%%%%%%

\usepackage{pdfpages}
% NOTE: To use, \includepdf[pages=-,pagecommand={},width=\textwidth]{file.pdf}

%%%%%%%%%%%%%%%%%%%%%%%%
% For Graphics/Images %
%%%%%%%%%%%%%%%%%%%%%%

\usepackage{graphicx}
\usepackage{wrapfig}
\usepackage{rotating}

%%%%%%%%%%%%%%%%%%%%%%%%%%
% For Tables and Graphs %
%%%%%%%%%%%%%%%%%%%%%%%%

\usepackage{booktabs} % For \toprule, \midrule and \bottomrule

\usepackage{siunitx} % Formats the units and values
\sisetup{
  round-mode          = places, % Rounds numbers..
  round-precision     = 2,      % ..to 2 places.
}

\usepackage{pgfplotstable} % Generates table from .csv
\usepackage{tikz}
\usepackage{pgfplots}
\pgfplotsset{compat=newest} % Allows to place legend below plot
\usepgfplotslibrary{units} % Enters in units nicely

\usepackage{float} % For figure spacing, allows float usage

%%%%%%%%%%%%%%%%%%%%%%%%%%%%%
% MATLAB CODE ORGANIZATION %
%%%%%%%%%%%%%%%%%%%%%%%%%%%

% This stuff allows one to easily include MATLAB code in the document
% without publishing from a MATLAB instance.

%% NOTE HOW TO USE:
%%% USE \mcode{} for inline
%%% use \begin{lstlisting}

\usepackage{listings}
\usepackage{xcolor}
\usepackage{textcomp}

\usepackage{hyperref}

\usepackage{setspace}
\doublespacing

%%%%%%%%%%%%%%%%%%%%%%%%%%%%%%%
% Custom Enumeration Options %
%%%%%%%%%%%%%%%%%%%%%%%%%%%%%

% This package allows one to make custom enumeration tabs.

%% EXAMPLE:
% \begin{enumerate}[label=(\alph*)]
% \item an apple
% \item a banana
% \item a carrot
% \item a durian
% \end{enumerate}

%% LABELS:
% \alph* : lowercase alphabet (i.e. a, b, c, ..)
% \Alph* : Uppercase alphabet (i.e. A, B, C, ..)
% \roman* : roman numerals (i.e. i, ii, iii, ..)

\usepackage{enumitem}

%%%%%%%%%%%%%%%
% Citations. %
%%%%%%%%%%%%%

\usepackage{cite}
\usepackage{url}

% NOTE (example usage): \cite{morley_scaroni_2017}
% Make a new file such as 'mybib.bib'
% Use the latexBibEntry snippet to create an entry.


 %%%%%%%%%%%%%%
% THE DOCUMENT %%%%%%%%%%%%%%%%%%%%%%%%%%%%%%%%%%%%%%%%%%%%%%%%%%%%%%%%%%%%%%%%
 %%%%%%%%%%%%%%

\begin{document}

%%%%%%%%%%%%%%%
% Title Page %
%%%%%%%%%%%%%

\pagenumbering{gobble}

\title{CSC 240: Final Project Proposal}
\author{The Classification of Cyberbullying Tweets}
\date{By: Sammy Haq}

% Comment the following out if you don't want a title page:
\maketitle
\newpage
%

%%%%%%%%%%%%%%%%%%%%%%
% Table of Contents %
%%%%%%%%%%%%%%%%%%%%

% Uncomment the following out if you want a Table of Contents:
%\tableofcontents
%\newpage
%

%%%%%%%%%%%%%%%%%%
% Document Body %
%%%%%%%%%%%%%%%%

% Comment this out if you don't want pages:
\pagenumbering{arabic}
%

\section*{Abstract}
As technology becomes more prevalent in our society and our interactions with
other people, cyberbullying becomes an increased problem. However, the internet
is expansive -- as a result, monitoring cyberbullying is a difficult task if
only done by human workers. Therefore, methods to automatically identify
cyberbullying via a more programatic approach, to at least ``filter'' possible
cyberbullying should be examined. After classifying cyberbullying tweets given in
the \href{https://www.kaggle.com/dataturks/dataset-for-detection-of-cybertrolls}
{Dataset for Detection of Cyber-Trolls} on Kaggle, we will then verify our
classifier's ``skill'' via K-Fold Cross Validation with $k = 10$. Limitations of
the classifier and further discussion will shortly follow.


%%%%%%%%%%%%%%%%%%%%%%%%%%%%%%%%%%%%%%%%%%%%%%%%%%%%%%%%%%%%%%%%%%%%%%%%%%%%%%%%
\section{Introduction} %%%%%%%%%%%%%%%%%%%%%%%%%%%%%%%%%%%%%%%%%%%%%%%%%%%%%%%%%
%%%%%%%%%%%%%%%%%%%%%%%%%%%%%%%%%%%%%%%%%%%%%%%%%%%%%%%%%%%%%%%%%%%%%%%%%%%%%%%%
Interactions have taken to cyberspace. From major politicians such as the
President of the United States to teenagers, the internet is the modern day
wild west. Due to this widespread use, interactions that may have previously
been in person can now be done online. This applies to workspace meetings such
as the use of Slack, to social interactions like those on Facebook and Twitter.

Unfortunately, malicious interactions also take place on the internet, such as
cyberbullying. Although these interactions can be policed by teachers and other
authoritative figures in reality, policing these sort of interactions on the
internet is almost non-existant. Furthermore, the internet is such an expansive
place that the implementation of human moderators on these large forums is
extremely expensive and uneffective. To aid moderators in their policing of
cyberbullying, developing a filter to help automatically detect the more
extreme cases of cyberbullying should be considered.

%%%%%%%%%%%%%%%%%%%%%%%%%%%%%%%%%%%%%%%%%%%%%%%%%%%%%%%%%%%%%%%%%%%%%%%%%%%%%%%%
\section{Research Objectives} %%%%%%%%%%%%%%%%%%%%%%%%%%%%%%%%%%%%%%%%%%%%%%%%%%
%%%%%%%%%%%%%%%%%%%%%%%%%%%%%%%%%%%%%%%%%%%%%%%%%%%%%%%%%%%%%%%%%%%%%%%%%%%%%%%%
This paper proposes the examination of the
\href{https://www.kaggle.com/dataturks/dataset-for-detection-of-cybertrolls}
{Dataset for Detection of Cyber-Trolls} found on Kaggle. A convolutional
neural network will be implemented on this dataset in order to classify the
data. Associative pattern mining is disregarded, as although it may be
interesting what phrases are used to cyberbully, using a less simplisitic
method such as a convolutional neural network will allow the creation of
hidden attributes that may be not considered by humans, and therefore allow
more expression in the model.

After successful implementation of the classifier, the classifier's ability to
successfully classify (known as its ``skill'') will be examined via the
k-fold cross validation method with a value of $k=10$. Then, the behaviour of
the classifier will be discussed and its viability in fitting the data.

%%%%%%%%%%%%%%%%%%%%%%%%%%%%%%%%%%%%%%%%%%%%%%%%%%%%%%%%%%%%%%%%%%%%%%%%%%%%%%%%
\section{General Approach} %%%%%%%%%%%%%%%%%%%%%%%%%%%%%%%%%%%%%%%%%%%%%%%%%%%%%
%%%%%%%%%%%%%%%%%%%%%%%%%%%%%%%%%%%%%%%%%%%%%%%%%%%%%%%%%%%%%%%%%%%%%%%%%%%%%%%%
In total, there are $20001$ items that the model will be trained upon. There
are two possible classifications: $0$ (which corresponds to non-cyber
aggressive) and $1$ (which implies that the tweet is cyber-aggressive). These
remarks are tweets hand-labeled by contributors on
\href{https://dataturks.com/projects/abhishek.narayanan/Dataset%20for%20Detection%20of%20Cyber-Trolls}
{Dataturks}.

A feed-forward neural network is an artificial neural network where the
connections are non-recursive and do not allow feedback. A convolutional
neural network (CNN) is a class of feed-forward artifical neural networks that
are based upon a specialized version of a multi-layer perceptron network.
The multi-layer perceptrons in convolutional neural networks require minimal
pre-processing, or manipulation of the raw data to be trained upon. This lack of
human interaction, as mentioned earlier, is a huge advantage as it does not
require the creation of handmade attributes to classify upon. This allows the
classifier to create its own attributes, free from human-bias.

Although convolutional neural networks are largely used in image processing,
research done by \href{https://dl.acm.org/citation.cfm?doid=1390156.1390177}
{Collobert and Weston (2008)} show promising results
regarding applying convolutional neural networks to natural language
processing.

The implmentation of the convolutional neural network classifier found in
the \href{https://scikit-learn.org/stable/index.html}{scikit-learn}
library will be used. The minimal data preprocessing required for this
task will be done with assistance from the \href{http://www.numpy.org/}{numpy}
and \href{https://pandas.pydata.org/}{pandas} libraries.

The
\href{https://scikit-learn.org/stable/index.html}{scikit-learn} library
will be generally used for most of the testing, as well -- namely, its
\href{https://scikit-learn.org/stable/modules/generated/sklearn.model_selection.
train_test_split.html}{train\_test\_split} class and
\href{https://scikit-learn.org/stable/modules/generated/sklearn.model_selection.KFold.html}
{K-Fold cross validator} will be used to train and test the classifier,
respectively.

\end{document}
